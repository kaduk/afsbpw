\documentclass{beamer}

\usepackage{graphicx}
\usepackage{beamerthemesplit}
\usepackage[normalem]{ulem}
\title{Updates from MIT Kerberos}
%\author{Benjamin Kaduk \\ \sout{\url{kaduk@mit.edu}} \url{bkaduk@akamai.com}}
\date{20 August, 2015}

\begin{document}

\AtBeginSection[]
{
    \begin{frame}
	\tableofcontents[currentsection]
    \end{frame}
}

\frame{\titlepage}

\section{While we were gone\ldots{}}

\begin{frame}[fragile]
\frametitle{Things you should know about but might have missed}
\begin{itemize}
\item{\verb+KRB5\_TRACE+}
\item{kadmin purgekeys for krbtgt rekeying}
\item{\verb+DIR:+ ccache type and collection-enabled ccaches}
\item{GSS acceptors can wildcard hostname part of host-based service}
\item{client keytabs}
\end{itemize}
\end{frame}

\begin{frame}[fragile]
\frametitle{krb5-1.12}
Freshly released for EAKC 2014, but a quick recap:
\begin{itemize}
\item{More plugin interfaces: aname-to-lname, kuserok, host-realm,
	default-realm}
\item{KDB policy records are more flexible; no refcounts $\rightarrow$
	better performance}
\item{Support principals with no long-term keys (e.g., for OTP/PKINIT)}
\item{KDC support for FAST OTP (RFC 6560)}
\item{Improvents to the \verb+KEYRING:+ cache, including collection support}
\item{AES-NI when available}
\item{Experimental KDC audit pluggable interface}
\end{itemize}
\end{frame}

\section{Latest Release, 1.13}

\begin{frame}
\frametitle{Schedule}
\begin{itemize}
\item{Shortened 10-month release cycle (1 year is normal)}
\item{Released on-schedule October 15, 2014}
\item{Should align better with OS releases (Fedora, Ubuntu, etc.)}
\item{(krb5-1.14 is expected in October 2015)}
\end{itemize}
\end{frame}

\begin{frame}[fragile]
\frametitle{krb5-1.13 features}
\begin{itemize}
\item{HTTP(S) transport --- MS-KKDCP HTTP proxy}
\item{Hierarchical iprop}
\item{Support for configuring GSS mechanisms via \verb+/etc/gss/mech.d/*.conf+}
\item{Support for SASL binds in the LDAP KDB backend}
\item{KDC listens on TCP by default}
\item{KCM: cache type for Heimdal (e.g., OS X) compatibility}
\item{Support for ulocked database dumps for the DB2 KDB backend, to allow
	the KDC and kadmind to continue processing requests during dumps}
\end{itemize}
\end{frame}

\begin{frame}
\frametitle{Late-breaking news in krb5-1.13}
\begin{itemize}
\item{SPNEGO improvements for out-of tree mechs (e.g., NTLM)}
\item{fix build against libressl}
\item{ksu cleanup (but maybe you shouldn't use ksu)}
\item{KDC logging works with redirected stderr}
\item{Incremental improvements to the replay cache performance and
	correctness}
\end{itemize}
\end{frame}

\begin{frame}
details of these features
\end{frame}

\begin{frame}
\frametitle{replay cache}
Some protocols can be designed to not need a replay cache (by using
an acceptor subkey or other key confirmation methods).

For safety and correctness, other protocols need a cache to detect and
avoid replay attacks.  MIT krb5 supplies such an implementation at the
library level, but the implementation is not very performant.

Is replay cache performance an issue for anyone here?  Please talk to us!
\end{frame}

\subsection{Security}

\begin{frame}[fragile]
\frametitle{Security Advisories}
MITKRB5-SA-2014-001:
\begin{itemize}
\item{CVE-2014-4345: Buffer overrun in kadmind with LDAP backend}
\item{\verb+cpw -keepold+ triggere miscounding of array size}
\end{itemize}
MITKRB5-SA-2015-001
\begin{itemize}
\item{CVE-2014-5352: gss\_process\_context\_token() incorrectly frees context}
\item{CVE-2014-9421: kadmind doubly frees partial deserialization results}
\item{CVE-2014-9422: kadmind incorrectly validates server principal name}
\item{CVE-2014-9423: libgssrpc server applications leak uninitialized bytes}
\end{itemize}
\end{frame}

\section{Coming soon\ldots{}}

\begin{frame}
\frametitle{Upcoming items from MIT Kerberos}
\begin{itemize}
\item{krb5-1.14 in October}
\item{KfW 4.1 expected \ldots sometime this year}
\end{itemize}
\end{frame}

\begin{frame}[fragile]
\frametitle{krb5-1.14 features}
\begin{itemize}
\item{CAMMAC}
\item{Authentication Indicator}
\item{Hopefully, a reporting-friendly dump format}
\item{gss\_acquire\_cred\_with\_password behavior change}
\item{make \verb+FILE:+ cache somewhat more efficient}
\item{Don't generate new des3 and arcfour keys by default}
\item{Option for site-specific error message wrapping, and include the
	\verb+FILE:+ ccache name in errors}
\item{Use Linux OFD locks when available}
\end{itemize}
\end{frame}
\begin{frame}[fragile]
\frametitle{krb5-1.14 features (continued)}
\begin{itemize}
\item{(developers only) note skipped tests in \verb+make check+ output}
\item{Incremental improvements to multi-hop preauthentication}
\item{document \verb+FILE:+ ccache and keytab file formats}
\item{Support 32-bit kvno keytab extensions}
\item{Log a notice when kadm5.acl fails to parse}
\item{Disallow principal renames with LDAP backend}
\item{Improvements to incremental database propogation}
\item{Limit use of ``old'' and ``wrong'' krb5 mechanism OIDs}
\item{Limit use of IAKERB}
\end{itemize}
\end{frame}

\begin{frame}
details on features
\end{frame}

\section{On the horizon}

\begin{frame}
\frametitle{Candidates for krb5-1.15}
\begin{itemize}
\item{SPAKE preauth}
	\begin{itemize}
	\item{forward secrecy in generated session keys}
	\item{option for two-factor as part of same exchange}
	\item{When configured properly, attacker can't tell which factor was
		wrong}
	\end{itemize}
\item{More progress on Python kerberos for testing?}
\item{More progress on moving DNS resolution off clients to the KDC}
\end{itemize}
\end{frame}

\section{KfW}

\begin{frame}
\frametitle{KfW 4.1 outline}
\begin{itemize}
\item{KfW 4.0 (based off krb5-1.10) was released in 2012}
\item{\sout{Time}past time for an update}
\item{Based off krb5-1.13}
\item{Waiting for feedback from testers to release}
\item{Please test KfW 4.1 beta 2!}
\end{itemize}
\end{frame}

\begin{frame}
\frametitle{KfW 4.1 features}
\begin{itemize}
\item{Improved support for MSLSA: ccache type}
\item{New library for ribbon interface, more accessible for screen readers}
\item{Registry key for default realm}
\item{Modernization in installer sources}
\item{All the features from krb5 1.11 through krb5 1.13}
\end{itemize}
\end{frame}

\section{A bit further off}

\begin{frame}
\frametitle{Long-term goals}
\begin{itemize}
\item{Stop relying on the DNS!}
\item{Let the KDC do name resolution; it can have a copy of the zone, or a
	secure path to the nameserver, or similar}
\item{PAD, akin to the MSFT PAC}
\item{Pluggable interface for kadmin ACLs}
\item{API or KCM-like credentials cache}
\item{much more}
\end{itemize}
\end{frame}

\begin{frame}
\frametitle{}
\Large{Thanks!}
\end{frame}

\end{document}
